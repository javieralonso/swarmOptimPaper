% !TEX root = ../swarmOptimTutorial.tex

%%%%%%%%%%%%%%%%%%%%%%%%%%%%%%%%%%%%%%%%%%%%%%%%%%%%%%%%%%%%%%%%%
% Intro
\section{Overview of the state of the art}\label{sec:4}

Next, we give an overview of the state of the art in optimization and optimal control methods for solving the multi-robot problems described in Section~\ref{sec:2}, leveraging on the tools of Section~\ref{sec:3}.

%%%%%%%%%%%%%%%%%%%%%%%%%%%%%%%%%%%%%%%%%%%%%%%%%%%%%
\subsection{Multi-robot motion planning}

\subsubsection{Collision avoidance}
A large body of decentralized collision avoidance approaches exist for multi-agent scenarios.
Optimal control strategies include~\cite{Hoffmann:2008vl}, which uses a rule-based switching control law derived using optimal control and reachability analysis for systems with second order dynamics and acceleration constraints.
Alternatively, decentralized nonlinear model predictive control using potential functions for collision avoidance~\cite{Shim:2003ih} has also been explored.
As an alternative to traditional control approaches, \cite{vandenberg09} presented a method where agents share the avoidance effort, solved via a Linear Program in velocity space. This method was later extended to agents with dynamics by~\cite{snape09, alonsomora10, AlonsoMora:2014kb} and~\cite{alonsomora2015auro}.

Centralized approaches have also been explored, where collision-free commands are computed for the whole team of robots simultaneously.
\cite{Raghunathan:2004ga} converts the centralized optimal control problem to a finite dimensional nonlinear program (NLP) by the use of collocation on finite elements and solves it by the use of an interior point algorithm.
Optimization methods also include a mixed integer linear program (MILP) with constraints in dynamics~\cite{Kuwata:2007vq}. Alternatively, \cite{alonsomora13icra} describes an optimization in velocity space with motion constraints conned as either a non-convex MIQP or its linearized QP version, which can be solved efficiently for large teams of agents.

\subsubsection{Global motion planning}
optimization-based approaches for computing global trajectories for teams of robots are typically centralized and computationally involved due to the non-convex nature of the problem. These approaches optimize over a set of intermediate states from the initial configuration to the goal configuration.
\cite{Mellinger:2012fi} described a Mixed Integer approach and applied it to navigating small teams of quadrotors. As an alternative,~\cite{Saha:2014vi} formulated the problem in discretized linear temporal logic. These methods provide global guarantees but scale poorly with the number of robots. \cite{Yu:2013ij} reduces the problem to planning on a graph and presents an optimal yet computationally efficient Integer linear program.

For computational efficiency, but still in continuous space, algorithms converging to a local optimum of the optimization have been proposed. These include, a Sequential Convex Program~\cite{augugliaro12} employed to compute collision-free trajectories for multiple UAVs in obstacle-free environments and an ADMM-based iterative message passing algorithm~\cite{Bento:2013td} which can be easily parallelized.

%%%%%%%%%%%%%%%%%%%%%%%%%%%%%%%%%%%%%%%%%%%%%%%%%%%%%
\subsection{Formation planning}

We subdivide the works into two categories. \emph{Formation control}, where works are concerned with achieving a target formation in an obstacle-free environment. And, \emph{formation planning} were works are concerned with planning the motion of a team of robots in formation in an environment with obstacles.

\subsubsection{Formation control}
Rigidity theory~\cite{Eren.Belhumeur.ea:02} and Morse theory~\cite{Anderson:11} have been successful in establishing (global) stability of a few specific formations, such as triangular formations~\cite{Cao.Morse.ea:11} and tetrahedral formations~\cite{Anderson.Yu.ea:10}.
Towards a broader range of formations,~\cite{Ji:2006tr} presented a leader follower approach with results on controllability and optimal control in obstacle-free environments. An alternative leader-based approach for distributed formation control, where robots minimize the deviation from predefined relative distances via a quadratic program, was presented by~\cite{Turpin:2012uh} and implemented in a team of quadrotors.
A generalized constrained and nonlinear MPC formulation with guaranteed stability has been detailed by~\cite{Dunbar:2002fh} for asymptotic stabilization of multiple vehicles to an equilibrium formation.

%, extension of their previous work on a model independent coordination strategy for formation control~\cite{Egerstedt:2001jr}.
For centralized systems, \cite{alonsomora12ijrr} described a method for pattern formation and reconfiguration where the robot positions are obtained via optimal coverage of the pattern. Recently,~\cite{Ritz:uc} achieved centralized coordination of a team of three quadrotors carrying a net by defining an optimal control law.

In order to avoid centralized decision-making or leader election, decentralized consensus between the agents must be achieved. For example~\cite{Montijano:2014bn} formulates a distributed quadratic optimization for planar formations.

\subsubsection{Formation planning}
Convex optimization frameworks for navigating in formation while avoiding collisions with obstacles include semidefinite programming~\cite{Derenick:2010cc} which considers only 2D circular obstacles, second order cone programming for formation reconfiguration~\cite{Derenick:2007wb} which considers circular formations and triangulates the free 2D space to compute the optimal motion in formation, centralized linear optimization~\cite{Karamouzas:wm} with velocity obstacles and distributed quadratic optimization~\cite{AlonsoMora:2015wi} in velocity space for manipulation of deformable objects. 

For increased richness of the solution set, a non-convex optimization can be directly formulated. For centralized off-line trajectory generation,~\cite{Kushleyev:2012wy} proposed to divide the robots into teams, plan trajectories for each team (with fixed formation) via a MIP~\cite{Mellinger:2012fi} and rely on local controllers for navigation of the individual robots within each team.
In contrast, a real-time method for centralized local navigation was described by~\cite{alonsomora15iros}, which formulates a constrained optimization with avoidance of dynamic obstacles and formation reconfiguration and solves it via a Sequential Convex Program which scales well with the number of robots.










