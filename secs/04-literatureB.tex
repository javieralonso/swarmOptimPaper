% !TEX root = ../swarmOptimTutorial.tex

%%%%%%%%%%%%%%%%%%%%%%%%%%%%%%%%%%%%%%%%%%%%%%%%%%%%%%%%%%%%%%%%%
% Intro
%%%%%%%%%%%%%%%%%%%%%%%%%%%%%%%%%%%%%%%%%%%%%%%%%%%%%
\subsection{Task assignment}
Task assignment is closely related to the problem defined by combinatorial optimization.
A good overview with taxonomy on task assignment and its link to optimization is given in~\cite{Gerkey:2004il}. The simplest and most common case for task assignment is that of single robot per task and single task per robot, or ST-SR. For this case and given a cost matrix with the cost of assigning each robot to each task, an optimal assignment can be found via the Hungarian algorithm~\cite{kuhn55}. Alternatively, a suboptimal solution within a specified bound can be obtained via the Auction algorithm~\cite{Bertsekas:1992wx}.

An instance of ST-SR is that of goal assignment for a homogeneous team of robots. For this, motion planning and assignment are typically decoupled, as in~\cite{alonsomora12ijrr} where a local planner is employed to drive the robots towards the goals.
Recently, the goal assignment for a team of robots has been combined with motion planning towards concurrent planning and assignment~\cite{Turpin:2014bu}.

The general formulation where there are possibly more robots than the number of tasks, and when the cost of assignment is related to the geographical distance between the task locations is studied within the framework of \emph{vehicle routing}. A good overview of the static scenario, i.e., when no new tasks during the execution of the mission, is provided in \cite{PT-DV:01}. In general, most of the available literature on
routing for robotic networks focuses on static environments and does not
properly account for scenarios in which dynamic, stochastic and adversarial
events take place. The problem of planning routes through service demands that arrive
\emph{during} a mission execution is known as the ``dynamic vehicle routing
problem'' (abbreviated as the DVR problem). There are two key differences between static and
dynamic vehicle routing problems. First, planning algorithms should
actually provide \emph{policies} (in contrast to pre-planned routes) that
prescribe how the routes should evolve as a function of those inputs that
evolve in real-time. Second, dynamic demands (i.e., demands that vary over
time) add \emph{queueing phenomena} to the combinatorial nature of vehicle
routing.  In such a dynamic setting, it is natural to focus on steady-state
performance instead of optimizing the performance for a single demand.
Additionally, system stability in terms of the number of waiting demands is
an issue to be addressed.
A joint algorithmic and queueing
approach to the design of cooperative control and task allocation
strategies for \emph{networks of uninhabited vehicles} required to operate
in dynamic and uncertain environments. This approach is based upon the
pioneering work of Bertsimas and Van Ryzin~\cite{Bertsimas.vanRyzin:91,Bertsimas.vanRyzin:93,Bertsimas.vanRyzin:93b}, who introduced queueing methods to solve the simplest DVR
problem (a vehicle moves along straight lines and visits demands whose time
of arrival, location and on-site service are stochastic; information about
demand location is communicated to the vehicle upon demand arrival); see
also the earlier related work~\cite{Psaraftis:88}. The power of algorithmic queueing theory stems from the wide
spectrum of aspects, critical to the routing of robotic networks, for which
it enables a rigorous study; specific examples taken from our work in the
past few years include complex models for the demands 
such as time
constraints
 \cite{Pavone.Bisnik.ea:MONE09, Pavone.Frazzoli:ICRA10}, service
priorities~\cite{Smith.Pavone:SIAM09}, and translating demands
\cite{Bopardikar.Smith.ea:10}, problems concerning robotic implementation such as
adaptive and decentralized algorithms~\cite{Pavone.Frazzoli.ea:11,
  Arsie.Savla.ea:TAC09}, complex vehicle dynamics~\cite{Savla.Frazzoli.ea:TAC08,
  Enright.Savla.ea:JGCD09}, limited sensing range~\cite{Enright.Frazzoli:06}, and team
forming \cite{Smith.Bullo:09}, mobility-on-demand vehicle fleets~\cite{Smith:2013fa}, and even integration of humans in the design
space \cite{Savla.Temple.ea:CDC08}.


%%%%%%%%%%%%%%%%%%%%%%%%%%%%%%%%%%%%%%%%%%%%%%%%%%%%%
\subsection{Surveillance and monitoring tasks}
\emph{Coverage control} is among the simplest known settings for surveillance, where the objective is to determine a partition of the environment among the robots, and waiting locations for the robots in their respective partitions so that the average travel time to the locations in the respective partitions in minimized. The average travel time is computed with respect to spatial distribution representing the likelihood of events at different locations. When the spatial distribution is known, then gradient descent algorithms along the lines of the alternating algorithm by Lloyd~\cite{Lloyd:82} have been shown to be distributed and converge to stationary points of the the multi-center cost function~\cite{cortes04}. When the distribution is unknown, then adaptive algorithms have also been proposed~\cite{Arsie.Savla.ea:TAC09,Schwager:2009fz}. The coverage control algorithms have also been extended to the case of vehicles with motion constraints~\cite{Savla.Frazzoli:WAFR08,Enright.Savla.ea:JGCD09}. 

The scenarios where the robots have relatively small sensor footprints and/or the dynamics in the environment requires persistent close range investigation by the robots, are studied under  \emph{persistent surveillance}. This problem has been extensively studied, e.g., in \cite{Huynh.Enright.ea:10,Smith.Schwager.ea:11,Smith.Schwager.ea:12,Enright.Frazzoli:12,Yu.Schwager.ea:14}.

%\subsubsection{Optimal coverage}
%An iterative optimization via the Centroidal Voronoi Tesselation was introduced by~\cite{cortes04} and later extended for decentralized multi-robot control by~\cite{Schwager:2009fz}.

%\subsubsection{Rendezvous/ Flocking/ Consensus} The basic idea is to achieve the alignment of states (position, heading angle, memory state) in a decentralized system. Key references include \cite{Cortes.Martinez.ea:06} and \cite{Jadbabaie.Lin.ea:03}.

