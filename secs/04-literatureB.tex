% !TEX root = ../swarmOptimTutorial.tex

%%%%%%%%%%%%%%%%%%%%%%%%%%%%%%%%%%%%%%%%%%%%%%%%%%%%%%%%%%%%%%%%%
% Intro
%%%%%%%%%%%%%%%%%%%%%%%%%%%%%%%%%%%%%%%%%%%%%%%%%%%%%
\subsection{Task assignment}
Task assignment is closely related to the problem defined by combinatorial optimization.
A good overview with taxonomy on task assignment and its link to optimization is given in~\cite{Gerkey:2004il}. The simplest and most common case for task assignment is that of single robot per task and single task per robot, or ST-SR. For this case both optimal~\cite{kuhn55} and suboptimal algorithms~\cite{Bertsekas:1992wx} exist. \ksmargin{is \cite{Bertsekas:1992wx} about task assignment ?}

The general formulation where there are possibly more robots than the number of tasks, and when the cost of assignment is related to the geographical distance between the task locations is studied within the framework of \emph{vehicle routing}. A good overview of the static scenario, i.e., when no new tasks during the execution of the mission, is provided in \cite{PT-DV:01}. In general, most of the available literature on
routing for robotic networks focuses on static environments and does not
properly account for scenarios in which dynamic, stochastic and adversarial
events take place. The problem of planning routes through service demands that arrive
\emph{during} a mission execution is known as the ``dynamic vehicle routing
problem'' (abbreviated as the DVR problem). There are two key differences between static and
dynamic vehicle routing problems. First, planning algorithms should
actually provide \emph{policies} (in contrast to pre-planned routes) that
prescribe how the routes should evolve as a function of those inputs that
evolve in real-time. Second, dynamic demands (i.e., demands that vary over
time) add \emph{queueing phenomena} to the combinatorial nature of vehicle
routing.  In such a dynamic setting, it is natural to focus on steady-state
performance instead of optimizing the performance for a single demand.
Additionally, system stability in terms of the number of waiting demands is
an issue to be addressed.
A joint algorithmic and queueing
approach to the design of cooperative control and task allocation
strategies for \emph{networks of uninhabited vehicles} required to operate
in dynamic and uncertain environments. This approach is based upon the
pioneering work of Bertsimas and Van Ryzin~\cite{Bertsimas.vanRyzin:91,Bertsimas.vanRyzin:93,Bertsimas.vanRyzin:93b}, who introduced queueing methods to solve the simplest DVR
problem (a vehicle moves along straight lines and visits demands whose time
of arrival, location and on-site service are stochastic; information about
demand location is communicated to the vehicle upon demand arrival); see
also the earlier related work~\cite{Psaraftis:88}. The power of algorithmic queueing theory stems from the wide
spectrum of aspects, critical to the routing of robotic networks, for which
it enables a rigorous study; specific examples taken from our work in the
past few years include complex models for the demands 
such as time
constraints
 \cite{Pavone.Bisnik.ea:MONE09, Pavone.Frazzoli:ICRA10}, service
priorities~\cite{Smith.Pavone:SIAM09}, and translating demands
\cite{Bopardikar.Smith.ea:10}, problems concerning robotic implementation such as
adaptive and decentralized algorithms~\cite{Pavone.Frazzoli.ea:11,
  Arsie.Savla.ea:TAC09}, complex vehicle dynamics~\cite{Savla.Frazzoli.ea:TAC08,
  Enright.Savla.ea:JGCD09}, limited sensing range~\cite{Enright.Frazzoli:06}, and team
forming \cite{Smith.Bullo:09}, mobility-on-demand vehicle fleets~\cite{Smith:2013fa}, and even integration of humans in the design
space \cite{Savla.Temple.ea:CDC08}.


ST-SR has been combined with motion planning towards concurrent planning and assignment~\cite{Turpin:2014bu} and with local collision avoidance and optimal coverage towards pattern formation with large teams of robots~\cite{alonsomora12ijrr}.


%%%%%%%%%%%%%%%%%%%%%%%%%%%%%%%%%%%%%%%%%%%%%%%%%%%%%
\subsection{Surveillance and monitoring tasks}

\subsubsection{Optimal coverage}
An iterative optimization via the Centroidal Voronoi Tesselation was introduced by~\cite{cortes04} and later extended for decentralized multi-robot control by~\cite{Schwager:2009fz}.

\subsubsection{Rendezvous/ Flocking/ Consensus} The basic idea is to achieve the alignment of states (position, heading angle, memory state) in a decentralized system. Key references include \cite{Cortes.Martinez.ea:06} and \cite{Jadbabaie.Lin.ea:03}.

